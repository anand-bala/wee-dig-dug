  \section{View}

  The \textbf{View} is responsible for rendering \textbf{Model} onto the GUI. It possessed routines that
  the \textbf{Model} uses to trigger updates to the GUI. This implementation was chosen as updates can
  be triggered as and when the \textbf{Model} is updates.

  \subsection{GUI (\textit{PuTTY} output) Operation}
  \begin{table}[H]
  \begin{tabular}{rl}
    \texttt{FILE}:         &\texttt{gui.s}  \\
    \texttt{PRIMARY CONTRIBUTOR}:    &\texttt{Anand Balakrishnan (anandbal)}
  \end{tabular}
  \end{table}

    The \textbf{View} is holds many strings with ANSI escape sequences. Some of these stored strings are used to do the following:

    \begin{itemize}
      \item Change location of cursor.
      \item Display game stats (time left, current refresh interval, level, high score, current score, etc).
      \item Print empty board (just walls).
    \end{itemize}

    The \textbf{View} also holds routines whose primary function is to use these strings and manipulate GUI.
    It exposes the following subroutines so as to allow the \textbf{Model} to trigger updates to GUI.

    \begin{itemize}
      \item \texttt{draw\_empty\_board}
      \item \texttt{populate\_board}
      \item \texttt{update\_board}
      \item \texttt{clear\_sprite}
    \end{itemize}

    \subsubsection{Draw Empty Board}
    
    \subsubsection{Populate GUI with Sand and Sprites}

    \subsubsection{Update GUI}

    \subsubsection{Clear a Sprite from GUI}



