  \section{Model}

  \begin{table}[H]
  \begin{tabular}{rl}
    \texttt{FILES}:         &\texttt{model.s}, \texttt{collisions.s} \\
    \texttt{WRITTEN BY}:    &\texttt{Anand Balakrishnan (anandbal)}
  \end{tabular}
  \end{table}

  The \textbf{Model} maintains the internal representation of the board and triggers \textbf{View} updates. It exposed routines that allows the
  \textbf{Controller} to trigger updates on the \textbf{Model}.

    \subsection{Implementation}

    The \textbf{Model} consists of an \emph{``array''} (created using the \texttt{FILL} directive)
    of size $19 \times 15$ bytes, each byte representing a grain of sand.

    The \textbf{Model} also consists of \texttt{DCD} tables to hold information of sprites. These tables are structured similar to a \texttt{struct}, see \textbf{Listing \ref{lst:sprite-struct}}.
    There are also staticaly defined regions of memory that keep track of the various states the game could possibly be in,
    for example, \texttt{PAUSE}, \texttt{GAME\_OVER}.
    The \textbf{Model} is also responsible for keeping track of other variables of the game,
    such as, \texttt{LEVEL}, \texttt{HIGH\_SCORE}, \texttt{CURRENT\_SCORE} and \texttt{TIME}.


    \begin{lstlisting}[caption={Structure for \texttt{SPRITE} data},label={lst:sprite-struct}]
SPRITE
	DCD X_POS	; Holds X coordinate of the sprite
	DCD Y_POS	; Holds Y coordinate of the sprite
	DCD LIVES	; Holds Number of lives the sprite has
	DCD DIRECTION	; Code for direction the sprite is moving/facing
	DCD OLD_X_POS	; Previous X coordinate of sprite
	DCD OLD_Y_POS	; Previous Y coordinate of sprite
	DCD ORIGINAL_X	; Original X position (to reset when respawning)
	DCD ORIGINAL_Y	; Original Y position (to reset when respawning)

    \end{lstlisting}

    \subsection{Operations}

    Operations that are defined by the \textbf{Model} are:

    \begin{itemize}
      \item Initialize and reset model.
      \item Move sprites and update entire model.
      \item Handle and detect collisions.
      \item Get if sand exists at given (x,y) coordinate on the board.
      \item Clear sand at given coordinate (x,y).
      \item Toggle game states (\texttt{BEGIN\_GAME},\texttt{PAUSE}, \texttt{GAME\_OVER}, \texttt{RUNNING}).
      \item Update individual sprites.
      \item Spawn sprites.
    \end{itemize}

    \subsubsection{Initialize and Reset Model}

    \subsubsection{Game States and Representation}

    \subsubsection{Update Model and Control Sprites}

    \subsubsection{Collision Detection}



